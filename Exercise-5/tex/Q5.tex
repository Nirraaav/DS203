\section*{Main Learnings}

\subsection*{Understanding Clustering Algorithms}

\textbf{K-Means Clustering:}
K-Means is a partition-based algorithm that aims to minimize the variance within each cluster. It is effective for datasets with well-separated clusters but struggles with clusters of varying densities and shapes. It performed excellently on \texttt{Clusters-5-v0.csv}, which had clear boundaries, but its performance decreased for \texttt{Clusters-5-v1.csv} and \texttt{Clusters-5-v2.csv} due to overlapping and mixed clusters.

\textbf{Agglomerative Clustering:}
Agglomerative Clustering is a hierarchical method that builds clusters by merging pairs of clusters. It does not assume a specific number of clusters and can handle various cluster shapes better than K-Means. It showed better performance on \texttt{Clusters-5-v1.csv} compared to K-Means but also faced challenges with \texttt{Clusters-5-v2.csv}, where the class mixing affected its ability to form meaningful clusters.

\textbf{DBSCAN:}
DBSCAN is a density-based algorithm that can identify clusters of arbitrary shape and is robust to noise. It excels in datasets with varying densities and noise, such as \texttt{Clusters-5-v2.csv}. However, it requires careful tuning of parameters, such as \texttt{eps} and \texttt{min\_samples}, to achieve optimal results. DBSCAN successfully identified core clusters in \texttt{Clusters-5-v2.csv} but struggled with \texttt{Clusters-5-v0.csv} where clusters are well-separated.

\subsection*{Evaluating Clustering Performance with Metrics}

\textbf{Silhouette Score:}
The Silhouette Score provides a measure of how similar a point is to its own cluster compared to other clusters. It highlighted the effectiveness of clustering in \texttt{Clusters-5-v0.csv}, where all algorithms performed well. For \texttt{Clusters-5-v1.csv}, the score dropped due to overlapping clusters, and for \texttt{Clusters-5-v2.csv}, it indicated poor clustering quality due to significant class mixing.

\textbf{Calinski-Harabasz Score:}
The Calinski-Harabasz Score measures cluster compactness and separation. High scores for \texttt{Clusters-5-v0.csv} confirmed well-defined clusters. For \texttt{Clusters-5-v1.csv}, scores were lower due to some class overlap, and the lowest scores for \texttt{Clusters-5-v2.csv} indicated poor cluster separation.

\textbf{Davies-Bouldin Score:}
The Davies-Bouldin Score assesses the average similarity between each cluster and its most similar cluster. Low scores for \texttt{Clusters-5-v0.csv} reflected distinct clusters, while higher scores for \texttt{Clusters-5-v1.csv} and \texttt{Clusters-5-v2.csv} indicated increased similarity between clusters, affecting clustering effectiveness.

\subsection*{Key Insights from Python Library \texttt{sklearn}}

\texttt{sklearn} provides a comprehensive suite of functions for implementing and evaluating clustering algorithms. Key learnings include:

\begin{itemize}
    \item The \texttt{KMeans} class for K-Means clustering, allowing the specification of the number of clusters and distance metrics.
    \item The \texttt{AgglomerativeClustering} class for hierarchical clustering with various linkage criteria.
    \item The \texttt{DBSCAN} class for density-based clustering, offering flexibility with parameter tuning.
    \item Functions like \texttt{silhouette\_score}, \texttt{calinski\_harabasz\_score}, and \texttt{davies\_bouldin\_score} for metric calculations, aiding in the assessment of clustering quality.
\end{itemize}

\subsection*{Conclusion}
The exercise provided valuable insights into the performance and characteristics of various clustering algorithms. Each algorithm has its strengths and weaknesses depending on the dataset characteristics. K-Means is best for well-separated clusters, Agglomerative Clustering handles various shapes, and DBSCAN is effective in the presence of noise and varying densities. Evaluating clustering performance with metrics such as Silhouette Score, Calinski-Harabasz Score, and Davies-Bouldin Score is crucial for understanding and improving clustering results. The use of \texttt{sklearn} functions facilitated a deeper understanding and practical application of these concepts.


\clearpage