\section*{Step 3}

\begin{custombox}[label={box:Q3}]{Step 3}
	Generate outputs by setting \textbf{degree = 1}, \textbf{degree = 3}, \textbf{degree = 6}, \textbf{degree = 10}, in the \verb|PolynomialFeatures| function used in \verb|E3.ipynb| and analyze them as follows:
	\begin{enumerate}[label=(\alph*)]
		\item Review the \verb|augmented_data.csv| file generated in each case and document your observations.
		\item Create an overall qualitative summary based on a review and analysis of the Figures generated.
		\item Summarize and explain the variations in the metrics \textbf{across regression methods for a given degree} (ie. a given set of polynomial features). Cover both, train and test, metrics, and compare them.
		\item Summarize and explain the variations in the metrics \textbf{across degrees for a given regression method}. Cover both, train, and test metrics, and compare them.
		\item When \textbf{degree = 1} which method(s) result in acceptable regression models? Why?
		\item When \textbf{degree = 6} which method(s) result in acceptable regression models? Why?
		\item As the value of degree is increased to 10 which regression methods show the most impact? Why?
		\item Why do non-parametric methods like \verb|KNN| and \verb|Decision Tree| based methods generate good results even without feature engineering?
		\item What are the limitations of the non-parametric methods?
		\item  Given the results, should LinearRegression be used at all? Why, when? Justify your answer.
	\end{enumerate}
\end{custombox}

\clearpage